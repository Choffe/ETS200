\begin{table}

\begin{tabular}{ | l |  p{7cm} |}
\hline
\emph{Uttryck}	 & \emph{Betydelse} \\ \hline \hline \\
Versionshanteringssystem (VCS) &	System som hanterar historik för källkod och dokument. \\ \hline 
 \\Repositorie (Repo) &	Server där projektets dokument, filer och historik lagras. \\ \hline 
 \\ Merge	& När man slår ihop två versioner av samma dokument eller fil. \\ \hline 
\\Branch & En kopia av projektet som utvecklas för sig själv, parallellt med projektet.  \\ \hline
\\Centraliserat VCS (CVCS) &	VCS där alla utvecklare kopplar upp sig mot ett gemensamt repositorie. \\ \hline
\\Decentralisert VCS (DVCS) &	VCS där alla utvecklare kan välja att koppla upp sig mot ett gemensamt repositorie eller att dela projektfiler mellan sig. \\ \hline
\\Concurrent Versions System (CVS) &	Är ett centraliserat VCS som idag inte utvecklas längre. \\ \hline
\\Subversion (SVN) &	System som bygger på CVS men som fortfarande utvecklas aktivt. \\ \hline
\\Git &	Är ett decentraliserat VCS. \\ \hline
\\Java &	Är ett programmeringsspråk som används i studien. \\ \hline
\\Commit &	Termen i CVS/SVN för när en utvecklare har ändrat i en fil och lägger upp ändringen i VCS. \\ \hline
\\Push &	Term för när en utvecklare som använder Git trycker ut sitt repositorie till andra utvecklare. \\ \hline
\\Update &	Används i CVS/SVN när en utevecklare hämtar hem de senaste versionerna av projektet. \\ \hline
\\Pull	 & Samma som Update för CVS/SVN men för Git. \\ \hline
\\eXtreme Programming (XP) &	Utvecklingsmetod framtagen av Kent Beck. \\ \hline
\\Story &	När en funktion ska utvecklas i XP skrivs denna som en story som beskriver vad som ska göras. \\ \hline
\\Task	& När en story delas upp i mindre problem blir dessa tasks. \\ \hline
\\Spike &	Egentid för att experimentera fram lösningar eller ideer. \\ \hline
\\Refaktorisering &	När kod förenklas och förbättras genom att minska komplexitet och göra saker mer tydliga. \\ \hline
\\Gate Keeper &	Vid användande av Git kan alla nya ändringar skickas till en person som granskar och därefter godkänner eller avslår ändringarna.  \\ \hline
\\JavaDoc &	Dokumentation som beskriver en Java-klass och alla dess publika metoder. \\ \hline
\\JUnit	& Program som kan användas för att skriva enhetstester för Java-program. \\ \hline
\\Trac	& En Wiki hemsida som gruppen använder för att strukturera sitt arbete. \\ \hline
\newpage
\\Fast-forward & Git uppdatera till en ny version med ändringar från bara ett ställe och inga konflikter. \\ \hline

\end{tabular}
\end{table}
