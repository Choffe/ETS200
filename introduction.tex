Open Source projekt är något som med åren har blivit väldigt populärt. För många kommersiella program finns idag open source alternativ. Webbläsare, ordbehandlare och programmeringsspråk är alla exempel på detta. Ett exempel på detta är Microsoft Internet Explorer som år 2002 användes av 85,8\% av alla internetbesökare~\cite{Browser} men som idag har fått lämna plats åt bland annat Mozilla Firefox och Google Chrome som båda är open source. Firefox och Chrome står nu tillsammans för 72,4\% av alla internetbesök och det ökar för varje månad.

För att kunna arbeta effektivt i ett open source projekt behövs en typ av versionshanteringssystem (VCS) som organiserar och för historik över projektfilerna. Vilket VCS är då vanligast inom open source. Microsoft har undersökt frågan i en enkät som över 1000 utvecklare besvarat~\cite{OpenSource}. Undersökningen visade att år 2011 var Git det mest populära verktyget oavsett vilken plattform som användes. Samma undersökning hade genomförts året innan och visade att Gits popularitet hade ökat kraftigt. Vad detta beror på är vad vi vill ta reda på i den här djupstudien.

När man läser till Civilingenjör i Datateknik på Lunds Tekniska Högskola går man en kurs som heter Programvaruutveckling i Grupp (PVG) där ett mjukvaruprojekt utvecklas av en grupp med 8-10 utvecklare under en sju veckors period. Som VCS har de kursansvariga valt att använda systemet Apache Subversion (SVN). Gruppen leds av två coacher som hjälper gruppen att komma framåt och utvecklas. I vår roll som coacher väljer vi att istället använda oss av Git som VCS och se hur bra det fungerar. Våra tidigare erfarenheter av VCS är med SVN så vår studie kommer gå ut på att först lära oss Git och sen försöka lära ut det till gruppen. Kan Git göra samma jobb som SVN lika bra eller kanske bättre? 

Studien är strukturerad med en kort förklaring av hur vår situation ser ut först följt av en beskrivning av hur gruppens kurs samspelar med coachernas kurs. Därefter följer en kort beskrivning av vad ett VCS bidrar med samt en översiktlig beskrivning av de system som behandlas i studien. Efter det går vi igenom hur studien har genomförts för att sen avslutas med en resultat- och slutsatsdel. För att lättare följa med och förstå texten i studien finns det i Sektion 6 en terminologitabell där begrepp förklaras kortfattat.